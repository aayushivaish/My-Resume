\documentclass[resmargin, 10pt]{res} % Use the res.cls style, the font size can be changed to 11pt or 12pt here

\usepackage{helvet}
\usepackage{tabto}
\usepackage{comment}
\usepackage{hyperref}
%\usepackage[top=1in, bottom=1in]{geometry}
%\usepackage{fullpage}

\setlength{\textwidth}{5.2in} % Text width of the document

\begin{document}

%----------------------------------------------------------------------------------------
%	NAME AND ADDRESS SECTION
%----------------------------------------------------------------------------------------

\moveleft.5\hoffset\centerline{\Large\bf Aayushi Vaish} % Your name at the top
 
\moveleft\hoffset\vbox{\hrule width\resumewidth height 1pt}\smallskip % Horizontal line after name; adjust line thickness by changing the '1pt'
 
\moveleft.5\hoffset\centerline{+91-7906248323}
\moveleft.5\hoffset\centerline{aayushi.vaish.2016@gmail.com}
\moveleft.5\hoffset\centerline{\url{https://aayushivaish.github.io/}}

%----------------------------------------------------------------------------------------

\begin{resume}

\section{Research \\Interests}

Rich experience as electronics and communications engineer with focus on creating advanced systems for spectroscopic analysis of extra-solar planets and finding methods to understand the conditions of the planets revolving around black holes.

\section{Education}

{\sl Bachelor of Technology in Electronics and Communications Engineering}  \hfill Aug. 2016 - Sep. 2020 \\
JSS Academy of Technical Education (Dr. A. P. J. Abdul Kalam Technical University), Noida, Uttar Pradesh \\
\tabto{4em}CGPA: 7.92/10.0

\section{Awards and Honors}
%Scholarships/Fellowships
% \begin{itemize}
Excellent performance in quiz competition on “Astronomy and Astrophysics” by Zenith Astronomy Club.\hfill 2020\\
Bronze Medal in Monochrome Sketching to raise awareness for COVID-19 by AIIMS New Delhi. \hfill 2020\\
Qualified the pre-final round of International Astronomy and Astrophysics Competition. \hfill 2020\\
% \end{itemize} 

% \section{Publications} 

% \hangindent=0.7cm Adviser, A.B., \textbf{your last name, initials}, Collaborator, S.E., ``Really cool paper title.'' \emph{The Astrophysical Journal Supplement Series} 47.1 (20XX). 


\section{Work\\Experiences}

{\sl Satellite Communications - Trainee at Ultra Tech Cements}\hfill June 2019 - Jan 2020
\vspace{1.2em}
\begin{itemize}\itemsep -2pt
\item Analyzed one-way data broadcasting and two-way interactive communications using radio waves as medium.
\item Implemented Very Small Aperture Terminal and VSAT Services.
\item Performed site survey to determine satellite Look Angle and the location for ODU with "look angle" clearance.

\end{itemize} 

{\sl Transmission Management Trainee at Bharat Sanchar Nigam Limited}\hfill June 2018 - December 2018
\vspace{1.2em}
\begin{itemize}\itemsep -2pt
\item Provided insights in deployment of Optical Fiber for faster communication.
\item Performed ground tests on various Optical Fiber Laying Methods.
\item Researched on the role of Optical Fiber in 5G communications. 
\end{itemize} 


% \section{Talks}
% ``A Markov Chain Monte Carlo analysis of mid infrared ULIRG spectra'' \hfill July 20XX \\
% Name of Conference\\
% University of XX, Baltimore, MD


% \section{Posters}
% ``The Environments of Supernovae in Nearby Galaxies''
% \vspace{0.5em}
% \begin{itemize} \itemsep -2pt
% \item January 20XX: Conference for Undergrad Women in Physics, University of ZZ
% \item September 20XX: Summer research symposium, University of YY
% \end{itemize}


\section{Projects\\Undertaken}
{\sl Design and Implementation of a 16-bit RISC Processor Using Verilog HDL}
\vspace{1.2em}
\begin{itemize}\itemsep -2pt
\item A 16-bit RISC Processor was designed and implemented using Verilog HDL with the help of XILINX 14.2 and ISIM simulator. The purpose of this project was to learn digital designing using XILINX software and to implement what I
have studied in my previous courses related to Computer Architecture and Digital Electronics.
\end{itemize}  
\vspace{4.7em}
{\sl Smart Attendance System for Employees using face detection and ML with Python}
\vspace{1.2em}
\begin{itemize}\itemsep -2pt
\item In this project initially the details and sample pictures of employees working in a sample organization were
stored. For each employee around 10 pictures were used to create training data model. Once the AI is trained
with new data model, if a face is recognized by it then attendance is marked for corresponding employee and
saved in a CSV file along with other necessary details.

\end{itemize} 

\section{Skills}
\begin{itemize} \itemsep -2pt
\item Programming languages: Python, C++
\item Libraries: OpenCV, Numpy, Pandas, TensorFlow
\item Operating systems: Mac OS, Linux, Windows
\item Software: LaTeX, Git
% \item Observing: 
% \begin{itemize}
% \item 5 nights as primary observer with a high speed (10 ms exposure) imager on the 1m Table Mountain telescope 
% \item Designed scheduling blocks for 10 hours of VLA continuum observations (VLA program 19B-047)
% \end{itemize}


\end{itemize}

% \section{Professional Memberships}
% American Astronomical Society (AAS) \\
% Phi Beta Kappa National Honor Society \\

% \section{Outreach}

% {\sl Social Media Director}, University of XX Astronomy Outreach \hfill Jan. 20XX - present
% \vspace{1mm}
% \begin{itemize} \itemsep -2pt
% \item Run social media accounts for University of XX Astronomy department
% \item Coordinate recording of public lectures
% \end{itemize}

% {\sl Volunteer}, University of XX stargazing nights \hfill Apr. 20XX - present
% \vspace{1mm}
% \begin{itemize} \itemsep -2pt
% \item Identify targets for monthly stargazing nights
% \item Set up telescopes and explain science of objects to members of the public
% \end{itemize}

% \section{Extracurricular Activities}
% {\sl Backpacking}
% \vspace{1mm}
% \begin{itemize} \itemsep -2pt
% \item I like to explore off trail in the local mountains in search of hidden waterfalls
% \end{itemize}

% {\sl The Big Bangers}: University of XX hip-hop dance crew
% \vspace{1mm}
% \begin{itemize} \itemsep -2pt
% \item Organized practices as Choreography Chair \hfill Fall 20XX
% \item Organized performances and recruitment as President \hfill Spring 20XX
% \end{itemize}

%----------------------------------------------------------------------------------------

\end{resume}
\end{document}
